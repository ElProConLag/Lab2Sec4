\documentclass[letter,12pt]{article}
\usepackage[paperheight=27.94cm,paperwidth=21.59cm,bindingoffset=0in,left=3cm,right=2.0cm, top=3.5cm,bottom=2.5cm, headheight=200pt, headsep=1.0\baselineskip]{geometry}
\usepackage{graphicx,lastpage}
\usepackage{upgreek}
\usepackage{censor}
\usepackage[spanish,es-tabla]{babel}
\usepackage{pdfpages}
\usepackage{tabularx}
\usepackage{graphicx}
\usepackage{adjustbox}
\usepackage{xcolor}
\usepackage{colortbl}
\usepackage{rotating}
\usepackage{multirow}
\usepackage[utf8]{inputenc}
\usepackage{float}
\usepackage{hyperref}

\renewcommand{\tablename}{Tabla}
\usepackage{fancyhdr}
\pagestyle{fancy}

\fancyhead[L]{}
%
\begin{document}
%
   \title{\Huge{Informe Laboratorio 2}}

   \author{\textbf{Sección x} \\  \\Alumno x \\ e-mail: alumno.contacto@mail.udp.cl}
          
   \date{Septiembre de 2025}

   \maketitle
   
   \tableofcontents
 
  \newpage
  

\section{Descripción de actividades}
Utilizando la aplicación web vulnerable DVWA

(Damn Vulnerable Web App - \href{https://github.com/digininja/DVWA}{https://github.com/digininja/DVWA} (Enlaces a un sitio externo.)) realice las siguientes actividades:


\begin{itemize}
    \item Despliegue la aplicación en su equipo utilizando docker. Detalle el procedimiento y explique los parámetros que utilizó.
    \item Utilice Burpsuite (https://portswigger.net/burp/communitydownload (Enlaces a un sitio externo.)) para realizar un ataque de fuerza bruta contra formulario ubicado en vulnerabilities/brute. Explique el proceso y obtenga al menos 2 pares de usuario/contraseña válidos. Muestre las diferencias observadas en burpsuite.
    \item Utilice la herramienta cURL, a partir del código obtenido de inspect elements de su navegador, para realizar un acceso válido y uno inválido al formulario ubicado en vulnerabilities/brute. Indique 4 diferencias entre la página que retorna el acceso válido y la página que retorna un acceso inválido.
    \item Utilice la herramienta Hydra para realizar un ataque de fuerza bruta contra formulario ubicado en vulnerabilities/brute. Explique el proceso y obtenga al menos 2 pares de usuario/contraseña válidos.
    \item Compare los paquetes generados por hydra, burpsuite y cURL. ¿Qué diferencias encontró? ¿Hay forma de detectar a qué herramienta corresponde cada paquete?

    \item Desarrolle un script en Python para realizar un ataque de fuerza bruta:

    \begin{itemize}
        \item Utilice la librería requests para interactuar con el formulario ubicado en vulnerabilities/brute y desarrollar su propio script de fuerza bruta en Python.
        El script debe realizar intentos de inicio de sesión probando una lista de combinaciones de usuario/contraseña.

        \item  Identifique y explique la cabecera HTTP que empleará para realizar el ataque de fuerza bruta.

        \item  Muestre el código y los resultados obtenidos (al menos 2 combinaciones válidas de usuario/contraseña).

        \item Compare el rendimiento de este script en Python con las herramientas Hydra, Burpsuite, y cURL en términos de velocidad y detección.
    \end{itemize}

    \item  Investigue y describa 4 métodos comunes para prevenir o mitigar ataques de fuerza bruta en aplicaciones web:

    \begin{itemize}
        \item Para cada método, explique su funcionamiento, destacando en qué escenarios es más eficaz.

    \end{itemize}


    
\end{itemize}

\section{Desarrollo de actividades según criterio de rúbrica}

\subsection{Levantamiento de docker para correr DVWA (dvwa)}

\subsection{Redirección de puertos en docker (dvwa)}

\subsection{Obtención de consulta a replicar (burp)}

\subsection{Identificación de campos a modificar (burp)}

\subsection{Obtención de diccionarios para el ataque (burp)}

\subsection{Obtención de al menos 2 pares (burp)}

\subsection{Obtención de código de inspect element (curl)}

\subsection{Utilización de curl por terminal (curl)}

\subsection{Demuestra 4 diferencias (curl)}

\subsection{Instalación y versión a utilizar (hydra)}

\subsection{Explicación de comando a utilizar (hydra)}

\subsection{Obtención de al menos 2 pares (hydra)}

\subsection{Explicación paquete curl (tráfico)}

\subsection{Explicación paquete burp (tráfico)}

\subsection{Explicación paquete hydra (tráfico)}

\subsection{Mención de las diferencias (tráfico)}

\subsection{Detección de SW (tráfico)}

\subsection{Interacción con el formulario (python)}

\subsection{Cabeceras HTTP (python)}

\subsection{Obtención de al menos 2 pares (python)}

\subsection{Comparación de rendimiento con Hydra, Burpsuite, y cURL (python)}

\newpage
\subsection{Demuestra 4 métodos de mitigación (investigación)}

% Please add the following required packages to your document preamble:
%\begin{table}[htbp]

\section*{Conclusiones y comentarios}

\end{document}
