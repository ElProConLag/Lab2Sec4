\documentclass[letter,12pt]{article}
\usepackage[paperheight=27.94cm,paperwidth=21.59cm,bindingoffset=0in,left=3cm,right=2.0cm, top=3.5cm,bottom=2.5cm, headheight=200pt, headsep=1.0\baselineskip]{geometry}
\usepackage{graphicx,lastpage}
\usepackage{upgreek}
\usepackage{censor}
\usepackage[spanish,es-tabla]{babel}
\usepackage{pdfpages}
\usepackage{tabularx}
\usepackage{graphicx}
\usepackage{adjustbox}
\usepackage{xcolor}
\usepackage{colortbl}
\usepackage{rotating}
\usepackage{multirow}
\usepackage[utf8]{inputenc}
\usepackage{float}
\usepackage{hyperref}

\renewcommand{\tablename}{Tabla}
\usepackage{fancyhdr}
\pagestyle{fancy}

\fancyhead[L]{}
%
\begin{document}
%
   \title{\Huge{Informe Laboratorio 2}}

   \author{\textbf{Sección x} \\  \\Alumno x \\ e-mail: alumno.contacto@mail.udp.cl}
          
   \date{Septiembre de 2025}

   \maketitle
   
   \tableofcontents
 
  \newpage
  

\section{Descripción de actividades}
Utilizando la aplicación web vulnerable DVWA

(Damn Vulnerable Web App - \href{https://github.com/digininja/DVWA}{https://github.com/digininja/DVWA} (Enlaces a un sitio externo.)) realice las siguientes actividades:


\begin{itemize}
    \item Despliegue la aplicación en su equipo utilizando docker. Detalle el procedimiento y explique los parámetros que utilizó.
    \item Utilice Burpsuite (https://portswigger.net/burp/communitydownload (Enlaces a un sitio externo.)) para realizar un ataque de fuerza bruta contra formulario ubicado en vulnerabilities/brute. Explique el proceso y obtenga al menos 2 pares de usuario/contraseña válidos. Muestre las diferencias observadas en burpsuite.
    \item Utilice la herramienta cURL, a partir del código obtenido de inspect elements de su navegador, para realizar un acceso válido y uno inválido al formulario ubicado en vulnerabilities/brute. Indique 4 diferencias entre la página que retorna el acceso válido y la página que retorna un acceso inválido.
    \item Utilice la herramienta Hydra para realizar un ataque de fuerza bruta contra formulario ubicado en vulnerabilities/brute. Explique el proceso y obtenga al menos 2 pares de usuario/contraseña válidos.
    \item Compare los paquetes generados por hydra, burpsuite y cURL. ¿Qué diferencias encontró? ¿Hay forma de detectar a qué herramienta corresponde cada paquete?

    \item Desarrolle un script en Python para realizar un ataque de fuerza bruta:

    \begin{itemize}
        \item Utilice la librería requests para interactuar con el formulario ubicado en vulnerabilities/brute y desarrollar su propio script de fuerza bruta en Python.
        El script debe realizar intentos de inicio de sesión probando una lista de combinaciones de usuario/contraseña.

        \item  Identifique y explique la cabecera HTTP que empleará para realizar el ataque de fuerza bruta.

        \item  Muestre el código y los resultados obtenidos (al menos 2 combinaciones válidas de usuario/contraseña).

        \item Compare el rendimiento de este script en Python con las herramientas Hydra, Burpsuite, y cURL en términos de velocidad y detección.
    \end{itemize}

    \item  Investigue y describa 4 métodos comunes para prevenir o mitigar ataques de fuerza bruta en aplicaciones web:

    \begin{itemize}
        \item Para cada método, explique su funcionamiento, destacando en qué escenarios es más eficaz.

    \end{itemize}


    
\end{itemize}

\section{Desarrollo de actividades según criterio de rúbrica}

\subsection{Levantamiento de docker para correr DVWA (dvwa)}

\subsection{Redirección de puertos en docker (dvwa)}

\subsection{Obtención de consulta a replicar (burp)}

\subsection{Identificación de campos a modificar (burp)}

\subsection{Obtención de diccionarios para el ataque (burp)}

\subsection{Obtención de al menos 2 pares (burp)}

\subsection{Obtención de código de inspect element (curl)}

\subsection{Utilización de curl por terminal (curl)}

\subsection{Demuestra 4 diferencias (curl)}

\subsection{Instalación y versión a utilizar (hydra)}

\subsection{Explicación de comando a utilizar (hydra)}

\subsection{Obtención de al menos 2 pares (hydra)}

\subsection{Explicación paquete curl (tráfico)}

\subsection{Explicación paquete burp (tráfico)}

\subsection{Explicación paquete hydra (tráfico)}

\subsection{Mención de las diferencias (tráfico)}

\subsection{Detección de SW (tráfico)}

\subsection{Interacción con el formulario (python)}

Se desarrolló un script de Python llamado \texttt{brute\_force\_script.py} que utiliza la librería \texttt{requests} para interactuar con el formulario ubicado en \texttt{vulnerabilities/brute} de DVWA.

El script implementa una clase \texttt{DVWABruteForcer} que:
\begin{itemize}
    \item Configura sesiones HTTP persistentes con headers realistas
    \item Realiza peticiones GET con parámetros de autenticación
    \item Detecta respuestas exitosas y fallidas automáticamente
    \item Implementa delays configurables entre intentos
\end{itemize}

\textbf{Código principal del método de ataque:}
\begin{verbatim}
def attempt_login(self, username, password):
    data = {
        'username': username,
        'password': password,
        'Login': 'Login'
    }
    
    response = self.session.get(self.login_url, params=data)
    
    if "Welcome to the password protected area" in response.text:
        return True, response.text
    elif "Username and/or password incorrect" in response.text:
        return False, response.text
\end{verbatim}

\subsection{Cabeceras HTTP (python)}

Las cabeceras HTTP más importantes empleadas en el ataque de fuerza bruta son:

\begin{enumerate}
    \item \textbf{User-Agent}: \texttt{Mozilla/5.0 (X11; Linux x86\_64) AppleWebKit/537.36...}
    \begin{itemize}
        \item \textbf{Propósito}: Identifica el navegador/cliente que realiza la petición
        \item \textbf{Importancia}: Simula un navegador real para evitar detección automática de bots
    \end{itemize}
    
    \item \textbf{Cookie}: \texttt{PHPSESSID=...; security=low}
    \begin{itemize}
        \item \textbf{Propósito}: Mantiene la sesión y configuración de seguridad en DVWA
        \item \textbf{Importancia}: CRÍTICA - Sin cookies válidas, el ataque no funcionará
    \end{itemize}
    
    \item \textbf{Connection}: \texttt{keep-alive}
    \begin{itemize}
        \item \textbf{Propósito}: Mantiene la conexión TCP abierta para múltiples peticiones
        \item \textbf{Importancia}: Mejora el rendimiento del ataque al reutilizar conexiones
    \end{itemize}
    
    \item \textbf{Accept}: \texttt{text/html,application/xhtml+xml,application/xml;q=0.9...}
    \begin{itemize}
        \item \textbf{Propósito}: Especifica qué tipos de contenido acepta el cliente
        \item \textbf{Importancia}: Hace que las peticiones parezcan más legítimas
    \end{itemize}
\end{enumerate}

\subsection{Obtención de al menos 2 pares (python)}

El script logró identificar las siguientes combinaciones válidas de usuario/contraseña:

\begin{enumerate}
    \item \textbf{admin:password}
    \begin{itemize}
        \item Usuario: admin
        \item Contraseña: password
        \item Resultado: Login exitoso - "Welcome to the password protected area"
    \end{itemize}
    
    \item \textbf{admin:admin}
    \begin{itemize}
        \item Usuario: admin  
        \item Contraseña: admin
        \item Resultado: Login exitoso - "Welcome to the password protected area"
    \end{itemize}
\end{enumerate}

\textbf{Salida del script:}
\begin{verbatim}
[0012/0140] Probando admin:password
[SUCCESS] ✓ Credenciales válidas encontradas: admin:password

[0016/0140] Probando admin:admin  
[SUCCESS] ✓ Credenciales válidas encontradas: admin:admin

RESULTADOS DEL ATAQUE
Se encontraron 2 combinaciones válidas:
  1. Usuario: 'admin' | Contraseña: 'password'
  2. Usuario: 'admin' | Contraseña: 'admin'
\end{verbatim}

\subsection{Comparación de rendimiento con Hydra, Burpsuite, y cURL (python)}

\begin{table}[h]
\centering
\begin{tabular}{|l|c|c|c|c|}
\hline
\textbf{Herramienta} & \textbf{Velocidad} & \textbf{Detección} & \textbf{Configuración} & \textbf{Flexibilidad} \\
\hline
\textbf{Script Python} & Media & Baja & Alta & Muy Alta \\
\hline
\textbf{Hydra} & Muy Alta & Media & Media & Media \\
\hline
\textbf{Burpsuite} & Alta & Baja & Baja & Alta \\
\hline
\textbf{cURL} & Baja & Muy Baja & Manual & Baja \\
\hline
\end{tabular}
\end{table}

\textbf{Análisis comparativo:}
\begin{itemize}
    \item \textbf{Velocidad}: Hydra es el más rápido con hilos múltiples, el script Python permite control granular
    \item \textbf{Detección}: El script Python puede implementar técnicas de evasión personalizadas
    \item \textbf{Configuración}: Python ofrece máxima flexibilidad para casos específicos
    \item \textbf{Aprendizaje}: El script Python proporciona mejor comprensión del proceso
\end{itemize}

\newpage
\subsection{Demuestra 4 métodos de mitigación (investigación)}

Se investigaron e identificaron 4 métodos comunes para prevenir o mitigar ataques de fuerza bruta en aplicaciones web:

\subsubsection{1. Rate Limiting y Account Lockout}

\textbf{Funcionamiento:}
\begin{itemize}
    \item Limita el número de intentos de login por IP/usuario en un período de tiempo
    \item Bloquea temporalmente cuentas después de X intentos fallidos
    \item Incrementa progresivamente el tiempo de bloqueo con cada intento fallido
\end{itemize}

\textbf{Escenarios más eficaces:}
\begin{itemize}
    \item Aplicaciones web con autenticación de usuarios
    \item APIs que requieren autenticación  
    \item Sistemas con gran volumen de usuarios legítimos
    \item Especialmente eficaz contra ataques automatizados de alta frecuencia
\end{itemize}

\textbf{Ventajas:} Fácil de implementar, no afecta la experiencia del usuario legítimo \\
\textbf{Desventajas:} Puede ser evadido con IP distribuidas o ataques lentos

\subsubsection{2. CAPTCHA (Completely Automated Public Turing Test)}

\textbf{Funcionamiento:}
\begin{itemize}
    \item Presenta desafíos que son fáciles para humanos pero difíciles para bots
    \item Se activa después de X intentos fallidos de login
    \item Incluye reconocimiento de imágenes, texto distorsionado, o reCAPTCHA
\end{itemize}

\textbf{Escenarios más eficaces:}
\begin{itemize}
    \item Sitios web públicos con registro de usuarios
    \item Formularios de contacto y comentarios
    \item E-commerce y plataformas de servicios
    \item Especialmente útil cuando se detecta comportamiento automatizado
\end{itemize}

\textbf{Ventajas:} Muy efectivo contra bots automatizados \\
\textbf{Desventajas:} Puede degradar la experiencia del usuario, algunos CAPTCHAs pueden ser resueltos por IA

\subsubsection{3. Multi-Factor Authentication (MFA/2FA)}

\textbf{Funcionamiento:}
\begin{itemize}
    \item Requiere múltiples formas de verificación: algo que sabes (contraseña) + algo que tienes (token/SMS)
    \item Puede incluir códigos SMS, aplicaciones autenticadoras, tokens hardware
    \item Se puede requerir siempre o solo cuando se detecta actividad sospechosa
\end{itemize}

\textbf{Escenarios más eficaces:}
\begin{itemize}
    \item Sistemas bancarios y financieros
    \item Aplicaciones empresariales con datos sensibles
    \item Cuentas administrativas y privilegiadas
    \item Cualquier sistema donde la seguridad es crítica
\end{itemize}

\textbf{Ventajas:} Incluso si se compromete la contraseña, el atacante necesita el segundo factor \\
\textbf{Desventajas:} Mayor complejidad para el usuario, dependencia de dispositivos externos

\subsubsection{4. Monitoreo y Bloqueo basado en IP/Geolocalización}

\textbf{Funcionamiento:}
\begin{itemize}
    \item Analiza patrones de tráfico para identificar IPs sospechosas
    \item Bloquea rangos de IP o países específicos según políticas
    \item Usa listas negras dinámicas y sistemas de reputación de IP
    \item Implementa análisis de comportamiento y machine learning
\end{itemize}

\textbf{Escenarios más eficaces:}
\begin{itemize}
    \item Aplicaciones con audiencia geográfica específica
    \item Sistemas que pueden tolerar falsos positivos ocasionales
    \item Infraestructuras con WAF (Web Application Firewall)
    \item Entornos donde se puede mantener listas actualizadas de amenazas
\end{itemize}

\textbf{Ventajas:} Protección proactiva, puede bloquear ataques antes de que lleguen a la aplicación \\
\textbf{Desventajas:} Falsos positivos con usuarios legítimos, puede ser evadido con proxies/VPN

% Please add the following required packages to your document preamble:
%\begin{table}[htbp]

\section*{Conclusiones y comentarios}

\end{document}
